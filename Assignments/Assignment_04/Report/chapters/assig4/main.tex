\chapter{Máquina rotativa: Aplicação com script LUA}
\label{cap4}
Neste trabalho é apresentado o emprego dos conceitos executados nas atividades anteriores para a simulação de uma máquina síncrona. A primeira seção introduz brevemente uma noção sobre máquinas síncronas. A segunda seção apresenta os materiais e métodos utilizados para o desenvolvimento do trabalho. A terceira seção apresenta os resultados obtidos. A quarta e última seção realiza as discussões e conclusões acerca deste trabalho.
%Neste trabalho é apresentado uma otimização do script .LUA desenvolvido na atividade descrita no capítulo \ref{chap2}, referente a um problema de circuitos magnéticos, que por sua vez for descrito no capítulo \ref{chap1}. A primeira seção introduz brevemente uma noção sobre o uso de funções em linguagem de programação e do acesso de scripts de diferentes arquivos, utilizando a linguagem computacional LUA. A segunda seção apresenta os materiais e métodos utilizados para o desenvolvimento do trabalho. A terceira seção apresenta os resultados obtidos. A quarta e última seção realiza as discussões e conclusões acerca deste trabalho.
%\newpage
%\pagebreak
\input chapters/assig4/1_intro
%\newpage
%\pagebreak
\input chapters/assig4/2_metod
%\newpage
%\pagebreak
\input chapters/assig4/3_resul
%\newpage
%\pagebreak
\input chapters/assig4/4_dico
