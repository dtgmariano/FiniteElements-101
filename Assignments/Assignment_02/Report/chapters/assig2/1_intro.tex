%---------------------------------------------------------------
%	UNIVERSIDADE FEDERAL DE UBERLÂNDIA
%	Faculdade de Engenharia Elétrica
%	Programa de Pós-Graduação em Engenharia Elétrica
%	Programa de Pós-Graduação em Engenharia Biomédica
%	Laboratório de Engenharia Biomédica
%	Dissertação de Mestrado
%	Capítulo 1: Introdução
%---------------------------------------------------------------
\section{Introdução}
\label{cap2:introducao}
\subsection{Lua script}
Lua é uma linguagem de script de multiparadigma, pequena, reflexiva e leve, projetada para expandir aplicações em geral, por ser uma linguagem extensível (que une partes de um programa feitas em mais de uma linguagem), para prototipagem e para ser embarcada em softwares complexos, como jogos. Assemelha-se com Python, Ruby e Icon, entre outras. Foi projetada e desenvolvida na PUC-Rio, no Brasil.\\
A linguagem de extensão LUA foi usada para adicionar recursos de processamento para o programa FEMM. A camada de interação pode executar scripts LUA, selecionado a opção "Open Lua Script" no menu de arquivos do programa. Outra possibilidade é a execução de comandos LUA inseridos em um console próprio para a linguagem LUA. Além das funções inerentes da linguagem, existem funções específicas para o aplicativo FEMM.
\subsection{Comandos para FEMM-Lua}

\begin{itemize}
  \item \textbf{clearconsole()}: Limpa a janela de saída;
  \item \textbf{newdocument(doctype)}: Cria um novo documento pré-processador;
  \item \textbf{print()}: Comando de saída para envio de mensagem;
  \item \textbf{quit()}: Fecha todos os documentos e sai da camada de interação;
\end{itemize}

Os comandos para adicionar/remover objetos utilizados são:
\begin{itemize}
  \item \textbf{mi\_addnode(x,y)}: Adiciona um nó na coordenada x,y.
  \item \textbf{mi\_addsegment(x1,y1,x2,y2)}: Adiciona um segmento de linha entre os nós localizados em (x1,y1) e (x2,y2).
  \item \textbf{mi\_addblocklabel(x,y)}: Adiciona um bloco localizado na posição (x,y).
\end{itemize}

\begin{itemize}
  \item \textbf{mi\_probdeb()}: definições do problema (Frequência, unidade, tipo do problema, precisão, entre outros.);
\end{itemize}
