%---------------------------------------------------------------
%	UNIVERSIDADE FEDERAL DE UBERLÂNDIA
%	Faculdade de Engenharia Elétrica
%	Programa de Pós-Graduação em Engenharia Elétrica
%	Laboratório de Engenharia Biomédica
%	Dissertação de Mestrado
%	Capítulo 2: Revisão Bibliográfica
%---------------------------------------------------------------

\section{Discussões e Conclusões}
O trabalho vigente apresentou uma abordagem de um problema de circuitos magnéticos utilizando os métodos dos elementos finitos. As soluções numéricas baseadas nos métodos dos elementos finitos são bem comuns e tornaram-se ferramentas indispensáveis para a realização de análises e projetos de máquinas elétricas. Essas técnicas podem ser empregas para o refinamento de análises analíticas. Ainda que soluções analíticas exatas não sejam possíveis de se obter na prática, não se pode desconsiderar a importância da análise analítica no que diz respeito a compreesnão dos princípios fundamentais e do desempenho básico de máquinas elétricas. O software FEMM possibilitou de forma prática a modelagem de um problema de domínio magnético, utilizando os métodos dos elementos finitos. Os arquivos gerados pelo programa assim como esse relatório encontra-se disponíveis para acesso/download em um repositório do GitHub, no endereço \url{https://github.com/dtgmariano/FiniteElements-101/tree/master/Assignment_01}.
