\section{Discussões e Conclusões}
A linguagem LUA possibilita o desenvolvimento de projetos de elementos finitos na ferramenta FEMM, utilizando linguagem computacional. Essa abordagem possibilita diversas vantagens no desenvolvimento de projetos de elementos finitos:
\begin{itemize}
  \item Padronização de desenvolvimento;
  \item Reutilização de códigos;
  \item Facilidade para realizar manutenção em projetos;
  \item Flexibilidade em adaptar um projeto para futuras expansões;
\end{itemize}
É importante ressaltar a importância das boas práticas de programação durante o desenvolvimento de projetos para o FEMM utilizando o LUA. Identação correta, nomes consistentes para variáveis, comentários, padrões de projeto são alguns dos aspectos essenciais para que o emprego da linguagem LUA em elementos finitos não seja comprometido. Os arquivos gerados pelo programa assim como esse relatório encontra-se disponíveis para acesso/download em um repositório do GitHub, no endereço \url{https://github.com/dtgmariano/FiniteElements-101/tree/master/Assignment_02}.
