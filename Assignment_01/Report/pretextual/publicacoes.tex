% -*- coding: utf-8 -*-

% File: publicacoes.tex
% Description: Master's tesis - publications
% Date: 23/07/2013
% Author: Fábio Henrique
% Last update: /07/2013

\newpage

%\thispagestyle{empty}   %Not counting in page enumeration

\begin{center}

\section*{\Large{Publicações}}

\end{center}

Os artigos listados abaixo são referentes aos trabalhos publicados que foram desenvolvidos ao longo do mestrado e referem-se ao tema da dissertação.
\vspace{1cm}

%\section*{\large Trabalhos completos apresentados e publicados em anais de congressos}
\begin{itemize}

\item \textbf{A. N. Silva}, M. B. Silva, I. A. Marques, E. L. M. Naves, and A. B. Soares, “Proposal of a real-time computational tool for the measurement of spasticity in stroke patients”, Biosignals and Biorobotics Conference (2014): Biosignals and Robotics for Better and Safer Living (BRC), 5th ISSNIP-IEEE, May 2014.

\item \textbf{A. N. Silva}, I. A. Marques, M. B. Silva, E. L. M. Naves, and A. B. Soares, “Metodologia para o estudo de variáveis que influenciam a medida do limiar do reflexo de estiramento tônico”, Anais do XXIV Congresso Brasileiro de Engenharia Biomédica, pp. 2413–2416, 2014.

\item M. B. Silva, I. A. Marques, \textbf{A. N. Silva}, E. T. Palomari and A. B. Soares, “Avaliação da espasticidade baseada no limiar do reflexo de estiramento tônico”, Anais do XXIV Congresso Brasileiro de Engenharia Biomédica, pp. 617–620, 2014.

\item I. A. Marques, M. B. Silva, \textbf{A. N. Silva}, E. L. M. Naves and A. B. Soares, “Avaliação da espasticidade baseada na detecção do limiar do reflexo de estiramento tônico em tempo real”, Anais do XXIV Congresso Brasileiro de
Engenharia Biomédica, pp. 1581–1584, 2014.

\end{itemize}

\newpage
Outros artigos publicados durante o mestrado encontram-se listados abaixo:

\begin{itemize}
\item \textbf{A. N. Silva}, K. L. Nogueira, M. B. Silva, A. Cardoso, E. A. Lamounier, and A. B. Soares, “A virtual electromyographic biofeedback environment for motor rehabilitation therapies”, 2013 ISSNIP Biosignals and Biorobotics Conference: Biosignals and Robotics for Better and Safer Living (BRC), pp. 1–4, 2013.

\item \textbf{A. N. Silva}, Y. Morere, E. L. M. Naves, A. A. R. De Sa, and A. B. Soares, “Virtual electric wheelchair controlled by electromyographic signals,” in ISSNIP Biosignals and Biorobotics Conference, BRC, 2013.

\item \textbf{A. N. Silva}, A. B. Soares, E. A. Lamounier, K. L. Nogueira, A. A. R. de Sá, ``Desenvolvimento de um sistema eletromiografia para controle de ambiente virtual de biofeedback aplicado à reabilitação motora'', Anais do XXIII Congresso Brasileiro de Engenharia Biomédica, 2012.

\item M. C. Melo, A. A. R. de Sa, \textbf{A. N. Silva}, and A. B. Soares, “Proposal of a com-
putational interface of biofeedback for rehabilitation of victims of stroke,” in 2013
ISSNIP Biosignals and Biorobotics Conference: Biosignals and Robotics for Bet-
ter and Safer Living (BRC), pp. 1–6, IEEE, Feb. 2013.

\item D. T. G. Mariano, A. M. Freitas, L. M. D. Luiz, \textbf{A. N. Silva}, P. Pierre, and E. L. M.
Naves, “An accelerometer-based human computer interface driving an alternative
communication system,” in 5th ISSNIP-IEEE Biosignals and Biorobotics Confe-
rence (2014): Biosignals and Robotics for Better and Safer Living (BRC), pp. 1–5,
IEEE, May 2014.
\end{itemize}